\documentclass{beamer}
\usetheme{AnnArbor}
%\usebeamercolor{beetle}
%\usebeamerfont{structurebold}

\usepackage{graphicx}
\usepackage{tabularx}
\usepackage{subfig}
\usepackage{natbib}
\usepackage{tikz}
\usepackage{algorithm, algpseudocode}

%\usepackage{caption}
%\captionsetup[figure]{justification=justified, singlelinecheck=false}


\title[CDT]{CDT: Algorithms and termination proofs}
\author{Pranav Kant Gaur}
\institute[BARC, India]{Computer Division, \newline Bhabha Atomic Research Centre, Mumbai, India}
\titlegraphic{\includegraphics[width=2cm, height=2cm]{figures/barc_logo.jpg}}
\date{}

\begin{document}

%%%%%%%%%%%%%%%%%%
\begin{frame}
  \titlepage
\end{frame}
%%%%%%%%%%%%%%%%%%

%%%%%%%%%%%%%%%%%%
\begin{frame}{Concepts}
\begin{enumerate}
    \item Segment types
    \begin{enumerate}
        \item Type 1:\newline
            If none of the endpoints of segment $AB$ have a pair of incident segments with angle less than 90 between them.
        \item Type 2:\newline
            If one of the endpoints of segment $AB$ has a pair of incident segments with angle less than 90 between them.
    \end{enumerate}
    \item Reference point:\newline
    A point which forms largest circumsphere with endpoints of a segment $AB$ amongst the set of points which lie inside diameter sphere $AB$? 
    \item Local feature size of a vertex $v$:\newline
    The radius of the smallest ball centered at v that intersects
two segments or vertices in PLC that do not intersect each other.
\end{enumerate}
\end{frame}
%%%%%%%%%%%%%%%%%%


%%%%%%%%%%%%%%%%%%
\begin{frame}
\frametitle{Segment recovery algorithm}
\begin{algorithm}[H]
\caption{Segment recovery}
\begin{algorithmic}[1]
\Procedure{DelaunizeSegments}{$inputPLC, DelaunayTriangulation$}
	\State \textbf{Input}: $D_0, X_0$
	\State \textbf{Output}: $D_1, X_1$
	\State \textbf{Initialize}: $D_1$ \gets $D_0$, $X_1$ \gets $X_0$
	\Repeat
	\State Form a queue \textit{Q} of missing segments in $D_1$;
	\While {$Q \ne \phi$}
	\State remove a segment $e_{i}e_{j}$ from \textit{Q};
	\State split $e_{i}e_{j}$ using rule 1, 2 or 3;
	\State update $D_1$, $X_1$;
	\EndWhile
	\Until{no segment of $X_1$ is missing from $D_1$}
\EndProcedure
\end{algorithmic}
\end{algorithm}
\end{frame}
%%%%%%%%%%%%%%%%%%


%%%%%%%%%%%%%%%%%%
\begin{frame}[allowframebreaks]
\frametitle{Segment splitting rules}
\begin{enumerate}
	\item	If segment $e_{i}e_{j}$ is of \textit{type-1}, $v = e_{i}e_{j}\cap S$, where $S(c, r)$ is the sphere, $p$ is reference point for $e_{i}e_{j}$,is defined by:
		\begin{algorithm}[H]
		\caption{Rule 1}
		\begin{algorithmic}[1]
		\Procedure{SegmentRecoveryRule1}{}
			\If{$|e_{i}p| < \frac{1}{2}|e_{i}e_{j}|$}
				\State $c \gets e_i, r \gets |e_{i}p|$
			\ElsIf{$|e_{j}p| < \frac{1}{2}|e_{i}e_{j}|$}
				\State $c \gets e_j, r \gets |e_{j}p|$
			
			\Else
				\State $c \gets $e_i$, r \gets \frac{1}{2}|e_{i}e_{j}|$
			\EndIf	
		\EndProcedure
		\end{algorithmic}
		\end{algorithm}  
	\framebreak
	\item  If $e_{i}e_j$ is \textit{type-2}, let $e_k = O(e_i)$, then $v = e_{k}e_j \cap S$, where $S(c, r)$ is the sphere defined by the \textit{reference point} $p$ of $v$ with : 
	\begin{algorithm}[H]
	\caption{Rule 2}
	\begin{algorithmic}[1]
	    \Procedure{SegmentRecoveryRule2}{}
	        \State $c \gets e_{k}, r \gets |e_{k}p|$
	        \If{$|ve_j| < |vp|$}
	            \State continue
	        \Else
	            \State Do not insert $v$, Goto Rule-3
	        \EndIf     
	    \EndProcedure
	\end{algorithmic}
	\end{algorithm}
    \framebreak
	\item If $e_{i}e_j$ is \textit{type-2}, let $e_k = O(e_i)$, and $v'$ is the rejected vertex by rule 2 then, $v = e_{k}e_j \cap S$, where $S(c, r)$ is the sphere defined by the reference point $p$ of $v$:
	\begin{algorithm}[H]
	\caption{Rule 3}
	\begin{algorithmic}[1]
	    \Procedure{SegmentRecoveryRule3}{}
	        \State $c \gets e_k$
	        \If{$|pv^{`}| < \frac{1}{2}|e_{i}v^{`}|$}
	            \State $r \gets |e_{k}e_{i}| + |e_{i}v^{`}| - |pv^{`}|$
	        \Else
	            \State $r \gets |e_{k}e_i| + \frac{1}{2}|e_{i}v^{`}|$
	        \EndIf
	    \EndProcedure
	\end{algorithmic}
	\end{algorithm}
\end{enumerate}
\end{frame}
%%%%%%%%%%%%%%%%%%

%%%%%%%%%%%%%%%%%%
\begin{frame}
\frametitle{Segment recovery: Termination proof}
\begin{itemize}
    \item If $e_{i}e_{j}$ is \textit{type-1}, \newline
    $|e_{i}e_{j}| \geq min\{lfs(e_i), lfs(e_j)\}$, \newline
    $lfs$ : local feature size
    \item If $e_{i}e_{j}$ is \textit{type-2} and $e_{k} = O(e_i)$, \newline
    $|e_{i}e_{j}| \geq \frac{1}{C}lfs(e_k)$, when $e_{i} = e_{k}$, \newline
    $|e_{i}e_{j}| \geq $lfs(e_k)$sin(\theta)$, when $e_{i} \ne e_{k}$ \newline
    $C$ : 2 times number of segments incident on $e_k$, \newline
    $\theta$ : smallest angle between these segments
\end{itemize}
\end{frame}
%%%%%%%%%%%%%%%%%%

%%%%%%%%%%%%%%%%%%
\begin{frame}{Facet recovery algorithm}
%\fontsize{2pt}{1.2}\selectfont
\begin{algorithm}[H]
\caption{Facet recovery}
\begin{algorithmic}[1]
\small  {
\Procedure{Recover facets}{$inputPLC, DelaunayTriangulation$}
	\State \textbf{Input}: $D_0, X_0$
	\State \textbf{Output}: $M$
	\State \textbf{Initialize}: $M_i$ \gets $D_0$
	\Repeat
	\State Form a queue \textit{Q} of missing facets in $M_i$;
	\While {$Q \ne \phi$}
	\State remove a facet $f$ from \textit{Q};
	\State form a missing region $\omega$ containing $f$;
	\State form cavities $C_1, C_2$ by formCavity procedure;
	\For{each $i \gets 1, 2$}
	    \State \Call{RetetrahedralizeCavity}{$C_i$}
	\EndFor
	\EndWhile
	\Until{no facet of $X_1$ is missing from $M$}
\EndProcedure
}
\end{algorithmic}
\end{algorithm}
\end{frame}
%%%%%%%%%%%%%%%%%%

%%%%%%%%%%%%%%%%%%

%%%%%%%%%%%%%%%%%%
\begin{frame}{Cavity verification algorithm}
\begin{algorithm}[H]
\caption{Cavity verification/expansion}
\begin{algorithmic}[1]
    \tiny{
    \Procedure{FormCavity}{$cdtMesh, inputPLC, cavity_i$}
    \State \textbf{Input}: $C_{i}, X, M_{k}$ 
    \State \textbf{Output}: $C_{i}^{`}$ (expanded cavity)
    \State \textbf{Initialization}: \State $V \gets Vertices(C)$
    \State Verify $C$, expand if necessary,
    \Repeat
    \State form a queue $Q$ of all \textit{non-strongly Delaunay} faces of $C$ in $V$
    \For{each face $\sigma \in Q$, $\sigma$ still in $C$}
        \State let $t$ be a neighbor tet holding $\sigma$;
        \State remove $\sigma$ from $C$;
        \For{each face $\delta t$ of t, $\delta t \ne \sigma$}
            \If{$\del t$ is a face of $C$}
                \State remove $\delta t$ from $C$;
            \Else
                \State add $\delta t$ into $C$; 
            \EndIf
        \EndFor
        \State Update $C$, $V$;
    \EndFor
    \Until{$Q \ne 0$}
    \EndProcedure
    }
\end{algorithmic}
\end{algorithm}
\end{frame}
%%%%%%%%%%%%%%%%%%

%%%%%%%%%%%%%%%%%%
\begin{frame}{Cavity retetrahedralization algorithm}
\begin{algorithm}[H]
\caption{Refill cavity}
\begin{algorithmic}[1]
    \Procedure{RetetrahedralizeCavity}{$cdtMesh, inputPLC, cavity$}
    \State \textbf{Input}: $X, M_{i}, C_{n}$
    \State \textbf{Output}: $M$
    \State \textbf{Initialization}: \State $M \gets M_{i}$
    \State construct Delaunay tetrahedralization $D_c$ of vertices of $C$;
    \State identify faces of $C$ is $D_c$, mark each face of $D_c$ to be \textit{inside} or \textit{outside} wrt. $C$;
    \State remove \textit{outside} tets from $D_c$, fill remaining tets into $C$;
    \EndProcedure
\end{algorithmic}
\end{algorithm}
\end{frame}
%%%%%%%%%%%%%%%%%%


%%%%%%%%%%%%%%%%%%
\begin{frame}
\frametitle{Facet recovery: Termination proof}
    \begin{itemize}
        \item Facet verification guarantees cavity reterahedralization will work since it ensures all involved facets are \textit{strongly Delaunay}.
        \item Issue in facet verification: \newline
        \begin{itemize}
            \item Cavity expansion terminates?
            \begin{itemize}
                \item Set $\psi$ of facets in $M_{i}$ which do not intersect any constraint facet \textit{stop} the expansion.
            \end{itemize}
            \item Missing facets due to expansion are recovered later?
            \begin{itemize}
                \item Removal of a constraint facet during expansion will result in cavity $C'$ which is \textit{completely} inside $C$
                \item $Volume(C') > 0$
            \end{itemize}
        \end{itemize}
    \end{itemize}
\end{frame}
%%%%%%%%%%%%%%%%%%

%%%%%%%%%%%%%%%%%%
\begin{frame}
	Thank You!!
\end{frame}
%%%%%%%%%%%%%%%%%%
\end{document}
