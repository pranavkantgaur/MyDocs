\documentclass{beamer}
\usetheme{AnnArbor}
%\usebeamercolor{beetle}
%\usebeamerfont{structurebold}

\usepackage{graphicx}
\usepackage{tabularx}
\usepackage{subfig}
\usepackage{natbib}
\usepackage{tikz}
\usepackage{algorithm, algpseudocode}

%\usepackage{caption}
%\captionsetup[figure]{justification=justified, singlelinecheck=false}

\title[CDT]{CDT: Algorithms and termination proofs}
\author{Pranav Kant Gaur}
\institute[BARC, India]{Computer Division, \newline Bhabha Atomic Research Centre, Mumbai, India}
\titlegraphic{\includegraphics[width=2cm, height=2cm]{figures/barc_logo.jpg}}
\date{}

\begin{document}

%%%%%%%%%%%%%%%%%%
\begin{frame}
  \titlepage
\end{frame}
%%%%%%%%%%%%%%%%%%

%%%%%%%%%%%%%%%%%%
\begin{frame}
\frametitle{Segment recovery algorithm}
\begin{algorithm}[H]
\caption{Segment recovery}\label{euclid}
\begin{algorithmic}[1]
\Procedure{DelaunizeSegments}{$inputPLC$}
	\State \textbf{Input}:$D_0, X_0$
	\State \textbf{Output}:$D_1, X_1$
	\State \textbf{Initialize}:
	\State $D_1$ \gets $D_0$, $X_1$ \gets $X_0$
	\Repeat
	\State Form a queue \textit{Q} of missing segments in $D_1$;
	\While {$Q \ne \phi$}
	\State remove a segment $e_{i}e_{j}$ from $Q$;
	\State split $e_{i}e_{j}$ using rule 1, 2 or 3;
	\State update $D_1$, $X_1$;
	\EndWhile
	\Until{no segment of $X_1$ is missing from $D_1$}
\EndProcedure
\end{algorithmic}
\end{algorithm}
\end{frame}
%%%%%%%%%%%%%%%%%%

\end{document}
