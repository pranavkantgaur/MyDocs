\documentclass{beamer}

\usetheme{Warsaw}
\usepackage{graphicx}
\usepackage{multimedia}
\usepackage{tabularx}
\usepackage{subfig}
\usepackage{amsmath}

\title{Multiprojector seamless tiled display}
\author{Pranav Kant Gaur}
\institute{Graphics and Visualization section}

\date{\today}

\begin{document}

\begin{frame}
\titlepage
\end{frame}

%//////////////////////////////////////////////////////////////////////////////////////////////////////////////////////////////////
\begin{frame}
\frametitle{Agenda}
\begin{enumerate}
\item The Problem 
\item Motivation
\item Existing solutions
\item Why not a commercial system?
\item Solution overview
\item System setup
\item Algorithm 
\item Contributions
\item Results
\item Issues
\end{enumerate}
\end{frame}
%//////////////////////////////////////////////////////////////////////////////////////////////////////////////////////////////////

%//////////////////////////////////////////////////////////////////////////////////////////////////////////////////////////////////
\begin{frame}
\frametitle{The Problem}
%Given an arbitrarily arranged grid of projectors projecting at arbitrary positions onto the screen, find the region in each projector image buffer where if the original image is mapped onto will result in a seamless rectangular projection region on the screen.\newline
%In short, 
%ADD UNALIGNED=> ALIGNED ??????????????????????
%Given an arbitrarily arranged grid of projectors projecting at arbitrary positions onto the screen, how to manipulate each projector image so that projected content is \textit{seamless} across the adjacent projectors. \textb{ADD FIGURE!!!!}
We want to create a large image on the projection screen by combining images projected by multiple projectors \textbf{ADD FIGURE!!!!}

\begin{figure}
\includegraphics[width=4.5cm,height=3cm]{figures/system_setup.jpg}
\caption{Multiprojector tiled display}
\end{figure}

\end{frame}
%//////////////////////////////////////////////////////////////////////////////////////////////////////////////////////////////////

%//////////////////////////////////////////////////////////////////////////////////////////////////////////////////////////////////
\begin{frame}
\frametitle{Motivation}
\begin{enumerate}
\item Why \textit{Tiled}?\newline
An image with spatial resolution higher than that \textit{perceivable} by human eye cannot be visualized without reducing its resolution(So actually \textit{high megapixels} is just a marketing strategy, always check \textit{pixels per inch} instead). We can achieve this by spatially \textit{streching} the content.
\item Why \textit{Multiprojector}?\newline
Seams of monitors used in our earlier Tiled display system were \textit{distracting}. Projectors do not pose such limitation.
\end{enumerate}
\end{frame}

%//////////////////////////////////////////////////////////////////////////////////////////////////////////////////////////////////

\begin{frame}{Existing solutions}
\begin{enumerate}
\item Manual 
\item Automatic
\end{enumerate}
\end{frame}

%//////////////////////////////////////////////////////////////////////////////////////////////////////////////////////////////////

\begin{frame}{Why not a commercial system?}
\end{frame}

%//////////////////////////////////////////////////////////////////////////////////////////////////////////////////////////////////

\begin{frame}[label=concept]
\frametitle{Solution overview}
\begin{enumerate}
\item Geometric alignment
\begin{enumerate}
\item For each projector, determine distortion introduced by it  in the projected image on screen.
\item Apply inverse of this distortion to the projector image to recover geometrically continuos projection across neigboring projectors.
\end{enumerate}
\item Edge blending
\begin{enumerate}
\item Determine overlapping region between neighbouring projectors
\item Attenuate intensity of pixels in that region for all overlapping projectors so that their overlap does not create \textit{bright} junction between them
\end{enumerate}
\end{enumerate}
\end{frame}

%//////////////////////////////////////////////////////////////////////////////////////////////////////////////////////////////////


%//////////////////////////////////////////////////////////////////////////////////////////////////////////////////////////////////
\begin{frame}
\frametitle{System setup}
Developed system has:
\begin{enumerate}
\item 3X3 grid of projectors
\item Rear projection screen
\item 1 digital camera
\item Workstations arranged in Master-slave configuration
\end{enumerate}

\begin{figure}
\centering
\begin{tabularx}{\linewidth}{@{}cXX@{}}
\begin{tabular}{c c}
\hspace{0.5cm}\subfloat[System setup]{\includegraphics[width=4.5cm,height=3cm]{figures/setup.jpg}} & 
\subfloat[Projector-array behind the projection screen]{\includegraphics[width=4.5cm,height=3cm]{figures/projs.jpg}} \\
\end{tabular}
\end{tabularx}
\end{figure}

\end{frame}

%//////////////////////////////////////////////////////////////////////////////////////////////////////////////////////////////////

\begin{frame}
\frametitle{Algorithm}
\framesubtitle{Compute screen to camera relation}
\begin{enumerate}
\item We want \hyperlink{concept}{\textit{Geometrically aligned}} and \hyperlink{concept}{\textit{seamless}} content on projection screen.
\item Camera is just a \textit{feedback device}.
\item All later calculations are performed in screen-coordinate system.
\end{enumerate}

\begin{figure}
\includegraphics[width=6cm, height = 4cm]{figures/debug_image_features.jpg}
\caption{View from camera}
\end{figure}
\end{frame}

%//////////////////////////////////////////////////////////////////////////////////////////////////////////////////////////////////

\begin{frame}
\frametitle{Algorithm(contd.): Remove projector distortion}
\framesubtitle{Project and detect features for each projector}
Detected features are mapped to screen cooridnate system.

\begin{figure}
\includegraphics[width=6cm, height=4cm]{figures/detected_corners.jpg}
\caption{Low exposure image of detected features for central projectors}
\end{figure}

\end{frame}

%//////////////////////////////////////////////////////////////////////////////////////////////////////////////////////////////////

\begin{frame}
\frametitle{Algorithm(contd.): Remove projector distortion}
\framesubtitle{Compute \textit{local} bounding boxes}
Normalized pair of $(coordinate_{\textit{original image}}, coordinate_{\textit{detected}})$ for each checkerboard corner gives the \textit{\hyperlink{concept}{distortion}} information.
\begin{figure}
\includegraphics[width=6cm,height=4cm]{figures/all_bboxes.jpg}
\caption{Boxes bounding the projection region of each projector}
\end{figure}

\end{frame}

%//////////////////////////////////////////////////////////////////////////////////////////////////////////////////////////////////

\begin{frame}
\frametitle{Algorithm(contd.): Geometric continuity}
\framesubtitle{Compute \textit{global} bounding box}
\begin{enumerate}
\item Addressing all local bounding boxes wrt. a common coordinate system. 
\item Global bounding box represents the original image to projected on the screen.
\item Helps in computing share of each projector in the original image.
\end{enumerate}

\begin{figure}
\includegraphics[width=6cm,height=4cm]{figures/mod_all_bboxes.jpg}
\end{figure}
\end{frame}

%//////////////////////////////////////////////////////////////////////////////////////////////////////////////////////////////////

\begin{frame}
\frametitle{Algorithm(contd.): Seamlessness}
\framesubtitle{Compute alpha map}
\begin{enumerate}
\item Compute \hyperlink{concept}{region of overlap} between adjacent projectors.
\item Attenute image intensity of each overlapping projector.
\end{enumerate}

\begin{figure}
\centering
\begin{tabularx}{\linewidth}{@{}cXX@{}}
\begin{tabular}{c c}
\hspace{0.5cm}\subfloat[Projection region]{\includegraphics[width=4.5cm,height=3cm]{figures/setup.jpg}} &
\subfloat[Corresponding attenuation map]{\includegraphics[width=4.5cm,height=3cm]{figures/projs.jpg}} \\
\end{tabular}
\end{tabularx}
\end{figure}

\end{frame}

%//////////////////////////////////////////////////////////////////////////////////////////////////////////////////////////////////

\begin{frame}
\frametitle{Contributions}
\begin{enumerate}
\item Line fitting detected corners.
\item Using \hyperlink{crossrat}{Cross-ratio invariant} to recover full projection region.
\end{enumerate}

\end{frame}

%//////////////////////////////////////////////////////////////////////////////////////////////////////////////////////////////////

\begin{frame}
\frametitle{Results}
\begin{enumerate}
\item Software
\begin{enumerate}
\item Written in C
\item Dependent on OpenCV(v2.4.1) and libgphoto2(v2.5.2)
\item Works on Ubuntu(12.04 LTS) and Scientific Linux(6.1)
\end{enumerate}      
\item Hardware
\begin{enumerate}
\item 3X3 grid of NEC 200X DLP projectors
\item 2.4mX1.8m acyilic glass based rear projection screen(from ScreenTech,Germany)
\item Canon Powershot G7 digital camera
\item 4 Workstations(1 master+ 3 slave) with ???\newline
Each slave controls rendering on a row of projectors. It recieves rendering information from master using \textit{Chromium} framework.
\end{enumerate}
\end{enumerate}
\end{frame}

%//////////////////////////////////////////////////////////////////////////////////////////////////////////////////////////////////

\begin{frame}
\frametitle{Results(contd.)}
\begin{enumerate}
\item Alignment procedure completes in 3-4 minutes as opposed to 30 mins. consumed in manual alignment approach
\item Proposed \textit{Cross ratio} based approach resulted in recovery of full projection region
\item Maximal misregistration between neighbouring projectors was around 2.5mm
SHOW FIGURE??????????????????
\item Paper selected for presentation at IEEE ICCCI-2014.
\end{enumerate}
\end{frame}

%//////////////////////////////////////////////////////////////////////////////////////////////////////////////////////////////////

\begin{frame}
\frametitle{Issues}
View independent color seamless is still an \textit{open} problem
EMBEDD A VIDEO WITH VOICE SHOWING THE PROBLEM IN THE DISPLAY
\movie[]{Color seamlessness problem}{capture.avi}
\end{frame}

%//////////////////////////////////////////////////////////////////////////////////////////////////////////////////////////////////

\begin{frame}
Thank you!!
\end{frame}
\appendix
\begin{frame}[label=crossrat]
Write formulae here!!!
\end{frame}

\end{document}
